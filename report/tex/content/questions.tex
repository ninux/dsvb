\section{Fragen}

\subsection{Q1}

Warum ist das Produkt zweier Werte im Q15-Format (\lstinline{short int}) in
der Funktion \lstinline{goertzel_filter_v0()} mit 14 Bit skaliert statt mit
15 Bit?

\subsubsection*{Antwort}
Im Normalfall kann das Produkt zweier Q15 Werte mit einem Bitshift um 15 Bit
skaliert werden.
\[ 
	( a_{Q_{15}} \cdot b_{Q_{15}}) \cdot \frac{1}{2^{15}} = c_{Q_{15}}
\]
Dies gilt jedoch nur wenn die beiden Faktoren nicht dem Maximalwert
entsprechen.
\[
	a = b = 2^{15}
\]
Als Regel kann man also formulieren, dass die Skalierung um so viele Bit
erfolgen muss, wie die Werte der Faktoren haben. Für den Spezialfall, dass
beide Faktoren den Maximalwert annehmen, muss um ein weiteres Bit geschoben
werden um einen Overflow zu verhindern.

Da im Code von \lstinline{goertzel_filter_v0()} nur um 14 Bit skaliert wird,
muss also schon einer der Faktoren mit einer Zweierpotenz skaliert sein.
Ein Blick in die Headerdatei \lstinline{goertzel.h} zeigt, dass die
Koeffizienten bereits als $\frac{a}{2}$ eingetragen sind. Somit wird das
Produkt schlussendlich doch mit $14+1=15$ Bit skaliert.
\[
	\left( \frac{a}{2^1} \cdot b \right) \cdot \frac{1}{2^{14}}
	= \frac{a b}{2^{(1+14)}}
	= \frac{a b}{2^{15}}
	= c
\]

\subsection{Q2}

Warum reduziert \lstinline{goertzel_filter_v1()} nicht nur den Speicherbedarf
sondern auch die Rechenzeit?

\subsubsection*{Antwort}
Die Anpassung des Algorithmus reduziert die Arraylänge von 3 auf 2 Elemente.
Dies ist eine offensichtliche Reduktion von Speicherbedarf. Die Rechenzeit
wird aber ebenfalls reduziert da ein Speicherzugriff entfällt. Dieser dürfte
sehr langsam sein, da dieser per Pointer referenziert wird und sicher nicht
nahe der CPU liegt (also kein Register). Die Berechnung, welche nun ohne
diesen Speicherzugriff erfolgt, wird zudem nur noch mit dem \lstinline{input}
Wert berechnet, welcher zudem per \emph{call-by-value} übergeben wird und
somit auf dem Stack liegt. Dies ist sicherlich massiv schneller als eine
Berechnung per Lese- und Schreibzugriff auf entfernten Speicher.

\subsection{Q3}

Ausgehend von der Formel (4.47) aus dem Skript soll gezeigt werden, dass die
Berechnungsmethode für die Signalleistung in der Funktion
\lstinline{goertzel_filter_v0} identisch ist zu jener welche in der Abbildung
4.17 gezeigt wird ausser mit dem Unterschied zum finalen Skalierungsfaktor 
$\frac{2}{N^2}$.

\subsubsection*{Antwort}
Formel aus dem Skript
\[
	P_k
	= 2 \left| \frac{X[k]}{N} \right|^2
	= \frac{2}{N^2} \left( \Re\{X[k]\}^2 + \Im\{X[k]\}^2 \right)
\]
Berechnung aus dem Code
\[
	\left( x[k-1] \right)^2
		+ \left( x[k-1] \right)^2
		- \left( a \cdot x[k-1] \cdot x[k-2] \right)
\]
Vom Parseval'schen Theorem wissen wir, dass für $n = N$ gilt
\[
	y_k[n] |_{n=N}
	= \sum_{i=0}^{N-1} x[i] W_N^{-k(N-1)}
	= X[k]
\]
Aus der Transferfunktion
\[
	y_k[n] = s[n] - W_N^k \cdot s[n-1]
\]
können wir nun die Leistungsformel mit den Werten aus dem Zeitbereich
formulieren zu
\[
	P_k = \frac{2}{N^2} \left(
		\left\lVert s[n] - W_N^k \cdot s[n-1] \right\rVert
	\right)^2
\]
In dieser Form können wir das $n$ beliebig schieben solange wir diese
Verschiebung auf allen Elementen machen. Somit können wir dies umschreiben zu
\[
	P_k = \frac{2}{N^2} \left(
		\left\lVert s[n-1] - W_N^k \cdot s[n-2] \right\rVert
	\right)^2
\]
Wenden wir nun die Operationen an so ergibt dies
\[
	P_k
	= \frac{2}{N^2} \left(
		s^2[n-1] + W_N^{2k} \cdot s^2[n-2]
		- 2 \cdot W_N^k \cdot s[n-1] \cdot s[n-2]
	\right)
\]
Mit der euler'schen Darstellung des \emph{twiddle factor}
\[
	W_N = e^{-j\frac{2 \pi}{N}}
\] 
ergibt sich die Rechnung zu
\[
	P_k
	= \frac{2}{N^2} \left(
		s^2[n-1] + e^{-j\frac{4 \pi k}{N}} \cdot s[n-2]
		- 2 e^{-j\frac{2 \pi k}{N}} \cdot s[n-1] \cdot s[n-2] 
	\right)
\]
Vergleicht man nun diese Formel mit jener welche im Code implementiert ist,
so zeigt sich, dass die $e$-Funktionen zu 1 werden müssen und $a=2$ ist. In
diesem Falle sind diese beiden Formeln gleich.
\[
	\begin{array}{r c l}
	P_k & = & \frac{2}{N^2} \left(
		s^2[n-1] +
		\underbrace{e^{-j\frac{4 \pi k}{N}}}_{1} \cdot s^2[n-2]
		- 2 \underbrace{e^{-j\frac{2 \pi k}{N}}}_{1} \cdot s[n-1] \cdot s[n-2] 
	\right) \\
	& \Rightarrow & \frac{2}{N^2} \left(
		s^2[n-1] + \cdot s^2[n-2] - 2 \cdot s[n-1] \cdot s[n-2] 
	\right)
	\end{array}
\]

\subsection{Q4}

Vergleichen Sie die Berechnungsmethoden der Versionen \lstinline{v0} und
\lstinline{v1} bezüglich deren Effizienz und numerischen Robustheit.
Welche Version ist vorzuziehen?
