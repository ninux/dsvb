\section{DSP Coding}

\subsection{ToDo -- 1}

\subsubsection{A}
Die Werte sind gegeben in der AppNote SPRA096A (p2, figure 1).

\subsubsection{B}
Der Wert für \lstinline{SCALING_FREQUENCY} ergibt sich aus
\[
	\Delta x = \frac{2 \pi f}{f_s} \\
	= 2 \cdot (2^{15} - 1 ) \cdot \frac{f}{f_s} \\
	= \underbrace{\frac{2 \cdot (2^{15} -1 )
		\cdot 2^{11}}{f_s}}_{\text{\lstinline{SCALING_FREQUENCY}}}
		\cdot \frac{f}{2^{11}} \qquad ,fs = 8\text{kHz}
\]
\[
	\Rightarrow \text{\lstinline{SCALING_FREQUENCY}} = 
	\frac{2 \cdot (2^{15} -1 ) \cdot 2^{11}}{8\text{kHz}}
		\approx 16777 \left[ \frac{1}{\text{Hz}} \right]
\]

\subsection{ToDo -- 2}
Die Werte sind gegeben in der AppNote SPRA096A (p7, table 2).

\subsection{ToDo -- 3}
Um das Array \lstinline{delay[]} auf zwei Elemente zu reduzieren kann der
Algorithmus nach folgendem Pseudocode implementiert werden:

\begin{lstlisting}
output   <- input + (a*delay[0]) - delay[1];
delay[1] <- delay[0];
delay[0] <- output;
\end{lstlisting}

\subsection{ToDo -- 4}


\subsection{ToDo -- 5}
