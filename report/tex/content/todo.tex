\section{DSP Coding}

\subsection{ToDo -- 1}

\subsubsection{A}
Die Werte sind gegeben in der AppNote SPRA096A (p2, figure 1).

\subsubsection{B}
Der Wert für \lstinline{SCALING_FREQUENCY} ergibt sich aus
\[
	\Delta x = \frac{2 \pi f}{f_s} \\
	= 2 \cdot (2^{15} - 1 ) \cdot \frac{f}{f_s} \\
	= \underbrace{\frac{2 \cdot (2^{15} -1 )
		\cdot 2^{11}}{f_s}}_{\text{\lstinline{SCALING_FREQUENCY}}}
		\cdot \frac{f}{2^{11}} \qquad ,fs = 8\text{kHz}
\]
\[
	\Rightarrow \text{\lstinline{SCALING_FREQUENCY}} = 
	\frac{2 \cdot (2^{15} -1 ) \cdot 2^{11}}{8\text{kHz}}
		\approx 16777 \left[ \frac{1}{\text{Hz}} \right]
\]

\subsection{ToDo -- 2}
Die Werte sind gegeben in der AppNote SPRA096A (p7, table 2).

\newpage
\subsection{ToDo -- 3}
Um das Array \lstinline{delay[]} auf zwei Elemente zu reduzieren kann der
Algorithmus nach folgendem Pseudocode implementiert werden:

\begin{lstlisting}
output   <- input + (a*delay[0]) - delay[1];
delay[1] <- delay[0];
delay[0] <- output;
\end{lstlisting}


\lstinputlisting[firstline=99, lastline=113, title=goertzel.c]{../src/goertzel.c}

Nun müssen allerdings auch die Indexe angepasst werden in abhängigen Funktionen.
\lstinputlisting[firstline=127, lastline=157, title=goertzel.c]{../src/goertzel.c}

\newpage
\subsection{ToDo -- 4}
Um die Koeffizienten zu berechnen müssen wir den Ausdruck für $W_N$ in
Real- und Imaginärteil zerlegen. Allgemein gilt die Beziehung
\[
	e^{j\varphi} =
		\underbrace{\cos(\varphi)}_{\Re\{e^{j\varphi}\}}
		+ j \underbrace{\sin(\varphi)}_{\Im\{e^{j\varphi}\}}
\]
Daraus leiten wir nun ab, wie die Koeffizienten zu berechnen sind für die
\emph{twiddle factors}
\[
	W_N
	= e^{-j\frac{2 \pi f}{f_s}}
	= \cos\left(\frac{2 \pi f}{f_s}\right)
		- j\sin\left(\frac{2 \pi f}{f_s}\right)
	\qquad , f_s = 8 \text{kHz}
\]

\begin{table}[h!]
	\centering
	\begin{tabular}{r r r}
		$f$	& $\Re(z) 2^{15}$	& $\Im(z) 2^{15}$ \\
		\hline
		697	& 27 980	& -17 055 \\
		770	& 26 956	& -18 631 \\
		825	& 26 127	& -19 777 \\
		941	& 24 219	& -22 072 \\
		1 209	& 19 073	& -26 645 \\
		1 336	& 16 325	& -28 412 \\
		1 477	& 13 085	& -30 042 \\
		1 633	& 9 315		& -31 416 \\
	\end{tabular}
	\caption{Twiddle Koeffizienten}
\end{table}

\lstinputlisting[firstline=69, lastline=94, title=goertzel.h]{../src/goertzel.h}

\subsection{ToDo -- 5}
\lstinputlisting[firstline=172, lastline=194, title=goertzel.c]{../src/goertzel.c}
